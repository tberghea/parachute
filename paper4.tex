

\documentclass[manuscript]{aastex}
\usepackage{amsmath}


\newcommand{\vdag}{(v)^\dagger}

%% You can insert a short comment on the title page using the command below.

\slugcomment{}

\shorttitle{A quadruply lensed quasar in PS1}
\shortauthors{C. Berghea et al.}

\begin{document}

\title{ Discovery of the first quadruple gravitationally lensed quasar candidate with Pan-STARRS }

\author{C. T. Berghea\altaffilmark{1}, G. Nelson\altaffilmark{1}, C. E. Rusu\altaffilmark{2}, C. R. Keeton\altaffilmark{3}, R. P. Dudik\altaffilmark{1}}

\altaffiltext{1}{U.S. Naval Observatory (USNO), 3450 Massachusetts Avenue NW, Washington, DC 20392, USA}
\altaffiltext{2}{Department of Physics, University of California, Davis, 1 Shields Avenue, CA 95616, USA}
\altaffiltext{3}{Rutgers, the State University of New Jersey, 136 Frelinghuysen Road, Piscataway, NJ 08854, USA}

\email{ciprian.t.berghea@navy.mil}

\begin{abstract}

We report the serendipitous discovery of the first gravitationally lensed quasar candidate from Pan-STARRS. The $grizy$ images reveal four point-like images with magnitudes between $14.9$ mag and $18.1$ mag. The colors of the point sources are similar, and they are more consistent with quasars than stars or galaxies. The lensing galaxy is detected in the $izy$ bands, with an inferred photometric redshift of $\sim 0.6$, lower than that of the point sources. We successfully model the system with a singular isothermal ellipsoid with shear, using the relative positions of the five objects as constraints. While the brightness ranking of the point sources is consistent with that of the model, we find discrepancies between the model-predicted and observed fluxes, likely due to microlensing by stars and millilensing due to dark matter substructure. In order to fully confirm the gravitational lens nature of this system, and add it to the small but growing number of the powerful probes of cosmology and astrophysics represented by quadruply lensed quasars, we further require spectroscopy and high-resolution imaging.

\end{abstract}

\keywords{black hole physics --- galaxies: individual (Holmberg IX) --- X-rays: binaries}

\section{INTRODUCTION}

In this letter we report the serendipitous discovery of a quadruply imaged gravitationally lensed quasar (quad lens) candidate from the Panoramic Survey Telescope and Rapid Response System (Pan-STARRS1, hereafter PS1) released images. To our knowledge, if proven with spectroscopic observations, this would be the first gravitational lens discovered in the PS1 data. There are only about two dozens known quad lenses on the whole sky \citep{masterlens}, and therefore increasing this sample is of high priority. We model the system to provide evidence in support of its lensing nature, and we conclude with the necessity of obtaining spectroscopic confirmation. Due to its position, this is not possible until the end of the year.

PS1 is a wide-field imaging system, with a 1.8 m telescope and 7.7 deg$^2$ field of view, located on the summit of Haleakala in the Hawaiian island of Maui. The first PS1 data was release in December 2016 including both images and photometry \citep[see][]{cha16}. The 1.4 Gpixel camera consists of 60 CCDs with pixel size of 0.256 arcsec \citep{ona08, ton09}. It uses five filters (g$_{P1}$, r$_{P1}$, i$_{P1}$, z$_{P1}$, y$_{P1}$, hereafter $grizy$), similar to the ones used by the Sloan Digital Sky Survey \citep[SDSS;][]{york00}. The largest survey PS1 performs is the 3$\pi$ survey, covering the entire sky north of $-30\deg$ declination. Given the large sky coverage, resolution and depth, PS1 is expected to contain nearly 2000 gravitationally lensed quasars, of which about 300 will be quad lensed quasars \citep{ogu10}.

NEED A PARAGRAPH ON THE OTHER QUADS OBSERVED SO FAR AND THEIR IMPORTANCE, AND REWRITE THIS PARAGRAPH. The importance of such quasars is illustrated by the wealth of science this lens has provided over the years.  Due to magnification, \citet{reis14} were able to measure the black hole spin in this quasar based on reflection features in X-ray spectra, which has important implications for studies of galaxy evolution. High resolution Hubbble images allowed \citet{bre08} to use source reconstruction to resolve the quasar host galaxy and reveal its structure. We hope this new lens will provide a similarly important results in the near future and in this letter we present results based solely on the PS1 data. 
 
We assume a cosmological model with $\Omega_M = 0.274$, $\Omega_L = 0.726$, and h = 0.71. All magnitudes are in the AB system. The structure of the paper is as follows:
In Section 2 we infer relative astrometry, photometry and photometric redshifts. In Section 3 we fit a mass model to the system. Discussions, conclusions and future works are presented in Section 4.



\section{MORPHOLOGICAL MODELING}


We used the stacked images from PS1 in all five filters in order to perform a morphological modeling. We present a close-up color image of the system in Fig.~\ref{lens}, which clearly shows four objects within $\lesssim 3.8\arcsec$ of each other. The three brightest of these are arranged in an arc-like configuration. We originally used a point spread function (PSF) constructed from stars in the PS1 field of view (FOV), but this left large residuals at the locations of the four objects, significant enough to affect the derived parameters. We suspected that this is due to spatial variation in the PSF across the PS1 FOV, and therefore adopted a technique of fitting the system with an analytical PSF, assumed to not change on the small scale represented by the system. We did this using Hostlens \citep{rus16}, a variant of ``glafic'' \citep{ogu10} which uses $\chi^2$ minimization in order to fit point sources as well as S{\'e}rsic \citep{ser63} profiles convolved with an analytical PSF. The analytical PSF is comprised of two concentric Moffat \citep{mof69} profiles, each one with its own full-width half maximum, ellipticity, orientation, and shape parameter. The two profiles are also characterized by a relative flux of one to the other. We successfully modeled the four objects as point sources convolved with the PSF, therefore showing that they are point-like. Our strategy was to run Hostlens starting from 100 different positions in the parameter space, select the best of the resulting models, and further run 10 Markov Chain Monte Carlo (MCMC) chains using the Metropolis�Hastings algorithm around it. The chains consist of one million steps, with an acceptance rate of $\sim0.3$, and we removed the first 1/5 of these (the ``burn-in'' steps). We also checked that the chains have converged, using the method of \citet{gel95}. We show MCMC-derived uncertainties between the various analytical parameters in Fig.~\ref{mcmc_hostlens}. In order to quantify additional uncertainties inherent in our method due to the non-analytical nature of the ``true'' PSF, we ran 100 simulations where we add noise of similar properties to the real images, on top of the best-fit model. Here, we created the best-fit model by using a PSF constructed from nearby stars, and we ran Hostlens on each of the 100 simulations, using an analytical PSF. As a final estimate of our uncertainties, we used the maximum between the MCMC-derived uncertainties and those from the simulations. We were able to obtain much improved residuals in all bands, and we show these in Fig.~\ref{hostlens}. 
 
We found conspicuous residual in the $i$, $z$ and $y$ bands, just north of the faint point source (Fig.~\ref{hostlens}, middle row). This is consistent with the discovery of a lensing galaxy, in support of the lensing nature of this system. Due to the relatively low signal to noise ratio and the proximity of the point sources, we were unable to fit the morphology of this galaxy with free parameters. The red colors, suggested by the non-detection in the bluer bands, suggest that it is a red, early-type galaxy. As a result, we modeled it with a S{\'e}rsic index of 4, typical of early-type galaxies. We checked that this produces a better fit than a S{\'e}rsic index of 1. For the S{\'e}rsic profile we used an axis ratio of 1 and we fixed the effective radius at $0.5\arcsec$. As the galaxy is most conspicuous in the $i$ and $y$ bands, we consider the most reliable estimate of the relative positions of all objects to be derived from the weighted average of the positions measured in these two bands, where we weight by the inverse of the measured uncertainties. The resulting astrometry, as well as photometry in each band, are shown in Table~1. We have orrected the magnitudes for galactic extinction using the maps of \citet{sch11}.

\section{PHOTOMETRIC REDSHIFTS}

chi from how many bands

{\bf This has to be changed to use Le Phare method }
To calculate photometric redshift we used the method of \citet{wu10}, which is based on Sloan Digital Sky survey (SDSS) colors. PS1 bands are similar but not identical to the SDSS bands and we used the corrections of \citep{fin16} to obtain magnitudes in the SDSS system. Folowing \citet{wu10} we minimize ${\chi}^2$ to obtain redshits for each of the sources. We plot $\chi^2_{\nu}$ in Figure~\ref{redshift} for each quasar image and also fitting all togehter. In the latter case we obtain the best fit at z $=$ 0.820$^{+0.006}_{-0.007}$.

The photometric, astrometric and redshift results are summarized in Table~1. 

Due to blending of A, B and C components, both Gaia \citep{gaia} and PS1 catalogs identified sources A and D only. 
PS1 psf magnitudes for A are in general smaller than what we obtained probably due to blending, but for D they are very similar to our uncorrected magnitudes: 18.599$\pm$0.013,  18.153$\pm$0.002,  18.079$\pm$0.007,  17.843$\pm$0.0093,  17.534$\pm$0.023. Gaia G magnitudes are 15.17 and 18.13 for A and D, respectively.
We will use the Gaia position for component D as absolute astrometric reference position for our lens: 26.7923069811, 46.5112731937. The errors for this position are 15.6 and 8.9 mas, respectively.

The lens is also bright in infrared, the unresolved WISE magnitudes for the lens are: 11.524$\pm$0.022, 10.434$\pm$0.020, 6.769$\pm$0.015 ,4.518$\pm$0.023  in the WISE bands $-$ W1, W2, W3 and W4, respectively. The WISE colors W1$-$W2$=$1.1 and W2$-$W3$=$3.7 match very well those of quasars at relatively low redshift \citep[e.g.][]{wu12}.

Finally, the lens is very likely a radio source. The NRAO VLA Sky Survey (NVSS) images show a source at the lens position, 2MASX J01471020+4630433 has a flux of 12.6 mJy at 1.4 GHz. The listed position for this source is only 3 arcsec away from the center of the lens. We are in the process of observing the lens with the Very Large Baseline Array (VLBA).


\section{LENS MODELING}


We used the gravlens software to model the gravitational lens \citep{kee01} using the averaged positions from the i and i bands, where the lensing galaxy is better detected. We chose not to use the fluxes in the modeling because it is very well known that they usually show large discrepancies from the models due to microlensing, millilensing or extinction. A good example is RX J1131-1231 which was mentioned in the introduction \citep{slu08}. Moreveover, fitting the fluxes could affect the results of models which do not account for small-scale structure or dust.  We therefore only used the observed fluxes to compare with the model predictions after the modeling was performed.

We use a Singular Isothermal Ellipsoid (SIE) with shear. The parameters for the best fit models are: $b = 1.932_{-0.011}^{+0.008}$, $x_0 = 0.284\pm0.006$, $y_0 = -2.301_{-0.049}^{0.052}$, $e_c = 0.163_{-0.070}^{+0.063}$, $e_s = -0.035_{-0.024}^{+0.023}$, $\gamma_c = 0.122_{-0.016}^{+0.015}$, $\gamma_s = 0.065\pm0.005$. The predicted positions and magnitudes are presented in Table~1. In Figure~\ref{model} we show the critical curve and the caustics. The $\chi^2_{\nu}$ is 0.8, with most dicrepancy coming from the position of the galaxy. We explored the range of models using MCMC methods. The predicted magnitude differences are shown in Table1~1 and a comparison with the observed values are shown in Figure~\ref{modelphot}. As expected we indeed find large discrepancies. 

 The model suggest that the two outer images should be similar in brightness.  This is interesting because there are some physical effects that can cause observed image brightnesses to differ from the predictions of simple lens models. Firstly light can be affected as it propagates through the main lens galaxy, for example by differential extinction.  This possibility is interesting in light of the color differences between images.  Propagation effects tend to be more important at optical wavelengths than at radio wavelengths; this is particularly true of extinction, although there can be scatter broadening at radio wavelengths. Secondly, the light can be affected by stars in the lens galaxy through microlensing.  This mainly affects optical wavelengths. Finally, the light can be affected by dark matter substructure in the lens galaxy through millilensing.  This could affect all wavelengths.

Measuring the flux ratios in radio with VLBA would provide valuable information that will help us test the lensing hypothesis and understand the various physical effects at play and better constrain the models 


\section{CONCLUSIONS AND FUTURE WORKS}

We have presented the first quadruply imaged gravitationally lensed quasar candidate in the Pan-STARRS1 Survey, discovered via visual inspection of the multi-band images. We find that the evidence supporting the gravitational lens nature of this system is overwhelming, and consists of the following:

\begin{itemize}
\item The presence of four point-like images within $\lesssim 3.8\arcsec$ of each other, with similar colors, completely consistent with quasars templates, but less consistent with stars and galaxies.
\item The detection of a red galaxy between the point source, such that the relative positions of all five objects are fully consistent with a well-known ``cusp'' configuration. We successfully reproduce this configuration using a SIE + shear model for the lensing galaxy. Further more, the brightness ranking of the point-like images is consistent with the ones predicted by the model.
 \item The photometric redshift analysis shows that the galaxy is of elliptical template, typical for lensing galaxies, and of redshift smaller than the point sources. For the inferred redshifts, the velocity dispersion and mass of the lensing galaxy are consistent with those of typical galaxies.
\end{itemize}

Nevertheless, we stop short of claiming that this is a confirmed lens until we can obtain spectroscopy of this system. Spectroscopy of the four point-like images would unequivocally demonstrate whether they are multiple images of the same background quasar, as well as allow to infer its redshift and physical properties. We also note that the system is accessible to ground based high-resolution observations at most adaptive optics-capable facilities in the Northern hemisphere, due to the proximity to a $R\sim12$ mag star $\sim18\arcsec$ away.

We note that this system is very similar to the well studied quad lens RX J1131-1231 \citep{slu03}, in terms of image separation and overall configuration. However, our system appears to have a larger source redshift and brighter images by about 2 magnitudes \citep{slu06}. \citet{slu08} also measure flux discrepancies for RX J1131-1231, and show that these flux discrepancies can be explained by microlensing. 



\acknowledgments

The Pan-STARRS1 Surveys (PS1) and the PS1 public science archive have been made possible through contributions by the Institute for Astronomy, the University of Hawaii, the Pan-STARRS Project Office, the Max-Planck Society and its participating institutes, the Max Planck Institute for Astronomy, Heidelberg and the Max Planck Institute for Extraterrestrial Physics, Garching, The Johns Hopkins University, Durham University, the University of Edinburgh, the Queen's University Belfast, the Harvard-Smithsonian Center for Astrophysics, the Las Cumbres Observatory Global Telescope Network Incorporated, the National Central University of Taiwan, the Space Telescope Science Institute, the National Aeronautics and Space Administration under Grant No. NNX08AR22G issued through the Planetary Science Division of the NASA Science Mission Directorate, the National Science Foundation Grant No. AST-1238877, the University of Maryland, Eotvos Lorand University (ELTE), the Los Alamos National Laboratory, and the Gordon and Betty Moore Foundation.
This work has made use of data from the European Space Agency (ESA)
mission {\it Gaia} (\url{https://www.cosmos.esa.int/gaia}), processed by
the {\it Gaia} Data Processing and Analysis Consortium (DPAC,
\url{https://www.cosmos.esa.int/web/gaia/dpac/consortium}). Funding
for the DPAC has been provided by national institutions, in particular
the institutions participating in the {\it Gaia} Multilateral Agreement.

\begin{thebibliography}{}


\bibitem[Bertin \& Arnouts(1996)]{ber96} Bertin, E., \& Arnouts, S.\ 1996, \aaps, 117, 393 
\bibitem[Brewer \& Lewis(2008)]{bre08} Brewer, B.~J., \& Lewis, G.~F.\ 2008, \mnras, 390, 39 

\bibitem[Chambers et al.(2016)]{cha16} Chambers, K.~C., Magnier, E.~A., Metcalfe, N., et al.\ 2016, arXiv:1612.05560 
\bibitem[Finkbeiner et al.(2016)]{fin16} Finkbeiner, D.~P., Schlafly, E.~F., Schlegel, D.~J., et al.\ 2016, \apj, 822, 66
\bibitem[Gaia Collaboration et al.(2016)]{gaia} Gaia Collaboration, Brown, A.~G.~A., Vallenari, A., et al.\ 2016, \aap, 595, A2 
\bibitem[Gelman(1995)]{gel95} Gelman A. et al., 1995, Bayesian Data Analysis. CRC Press, Boca Raton, FL
\bibitem[Keeton(2001)]{kee01} Keeton, C.~R.\ 2001, arXiv:astro-ph/0102340 

\bibitem[Moffat (1969)]{mof69} Moffat, A. F. J.\ 1969, \aap, 3, 455

%% \bibitem[Mateos et al.(2012)]{mat12} Mateos, S., Alonso-Herrero, A., Carrera, F.~J., et al.\ 2012, \mnras, 426, 3271 
\bibitem[Oguri \& Marshall(2010)]{ogu10} Oguri, M., \& Marshall, P.~J.\ 2010, \mnras, 405, 2579 
\bibitem[Oguri (2010)]{ogu10} Oguri M., 2010, PASJ, 62, 1017 

\bibitem[Onaka \& al.(2008)]{ona08} Onaka P., Tonry J.~L., Isani S., Lee A., Uyeshiro R., Rae C., Robertson L., Ching G., Proc.\ 2008, \procspie, 7014, 12

\bibitem[Peng et al.(2010)]{peng10} Peng, C.~Y., Ho, L.~C., Impey, C.~D., \& Rix, H.-W.\ 2010, \aj, 139, 2097 
\bibitem[Reis et al.(2014)]{reis14} Reis, R.~C., Reynolds, M.~T., Miller, J.~M., \& Walton, D.~J.\ 2014, \nat, 507, 207 
\bibitem[Rusu et al.(2016)]{rus16} Rusu C.~E., et al., 2016, MNRAS, 458, 2

\bibitem[Schlafly \& Finkbeiner(2011)]{sch11} Schlafly, E.~F., \& Finkbeiner, D.~P.\ 2011, \apj, 737, 103 
\bibitem[S{\'e}rsic (1963)]{ser63} S{\'e}rsic J.~L., 1963, BAAA, 6, 41 
\bibitem[Sluse et al.(2003)]{slu03} Sluse, D., Surdej, J., Claeskens, J.-F., et al.\ 2003, \aap, 406, L43 
\bibitem[Sluse et al.(2006)]{slu06} Sluse, D., Claeskens, J.-F., Altieri, B., et al.\ 2006, \aap, 449, 539 
\bibitem[Sluse et al.(2008)]{slu08} Sluse, D., Eigenbrod, A., Courbin, F., et al.\ 2008, Manchester Microlensing Conference, 20 

\bibitem[Tonry \& Onaka(2008)]{ton09} Tonry J., Onaka P.\ 2009, in Ryan S., ed., 
Proceedings of the Advanced Maui Optical and Space Surveillance Technologies Conference.
The Maui Economic Development Board, Kihei, HI, p. E40

\bibitem[Wu \& Jia(2010)]{wu10} Wu, X.-B., \& Jia, Z.\ 2010, \mnras, 406, 1583 

\bibitem[Wu et al.(2012)]{wu12} Wu, X.-B., Hao, G., Jia, Z., Zhang, Y., \& Peng, N.\ 2012, \aj, 144, 49 
\bibitem[York et al.(2000)]{york00} York, D.~G., Adelman, J., Anderson, J.~E., Jr., et al.\ 2000, \aj, 120, 1579 


\clearpage

%%%%%%%%%%%%%%%%%%%%%%%%%%%%%%%%%%%%%%%%%%%%%%%%%%%%%%%%%%%%%%%%%%%%%%%%%%%%%%%%%%%%%%%%%%%


\end{thebibliography}


%%%%%%%%%%%%%%%%%%%%%%%%%%%%%%%%%%%%%%%%%%%%%%%%%%%%%%%%%%%%%%%%%%%%%%%%%%%%%%%%%%%%%%%%%%%

\begin{deluxetable}{l|ccccc}
\tablecolumns{5}  
\tablewidth{0pt} 
\tabletypesize{\tiny}
\setlength{\tabcolsep}{0.05in}
\tablenum{1} %\\               
\tablecaption{Relative astrometry and photometry}   
\tablehead{
\colhead{Property} & \colhead{A} & \colhead{B} & \colhead{C} & \colhead{D} & \colhead{G}\\
}
\startdata

\tableline
\multicolumn{4}{l}{\tiny {Measurements}}\\
\tableline

g & 15.60$\pm$0.01 &  15.72$\pm$0.01 &  16.45$\pm$0.02 & 18.09$\pm$0.01 & $\cdots$ \\
r & 15.40$\pm$0.01 &  15.55$\pm$0.01 &  16.21$\pm$0.01 & 17.74$\pm$0.01 & $\cdots$ \\
i & 15.36$\pm$0.01 &  15.57$\pm$0.02 &  16.15$\pm$0.02 & 17.74$\pm$0.02 & 19.50$\pm$0.20 \\
z & 15.23$\pm$0.03 &  15.50$\pm$0.05 &  16.02$\pm$0.01 & 17.68$\pm$0.03 & 18.95$\pm$0.13 \\
y & 14.92$\pm$0.01 &  15.23$\pm$0.02 & 15.76$\pm$0.02 & 17.36$\pm$0.02 & 19.20$\pm$0.24 \\
%$\Delta$i & 0.0 & 0.21 & 0.79 & 2.38 & 4.14 \\
$\Delta$x & 0.000$\pm$0.004 & -1.185$\pm$0.004 & 1.271$\pm$0.005 & 0.410$\pm$0.004 & 0.240$\pm$0.050 \\
$\Delta$y & 0.000$\pm$0.004 & -0.441$\pm$0.004 & -0.074$\pm$0.004 & -3.310$\pm$ 0.004 & -2.310$\pm$0.025 \\
%photoz & 0.818$^{+0.007}_{-0.009}$ & 0.732$^{+0.007}_{-0.006}$ & 0.912$^{+0.008}_{-0.009}$ & 0.914$^{+0.007}_{-0.009}$  \\

\tableline
\multicolumn{3}{l}{\tiny {Model prediction}}\\
\tableline

$\Delta$m & 0.0 &  0.580$^{+0.17}_{-0.020}$ & 0.626$^{+0.005}_{-0.009}$ & 3.55$\pm$0.06 \\


\enddata


\tablecomments{
The first four columns are the four quasar images and the fifth column is the lens.
The first five rows are magnitude measurements derived using Hostlens, corrected for extinction. $\Delta$x is positive towards west, and $\Delta$x towards north.
The next two raws are the relative positions of each object relative to A. We assume a pixel scale of $0.256\arcsec$. 
%The following three rows are relative i magnitudes and relative positions in arcseconds averaged between the i and y bands. 
%The next row shows the photometric redshift estimates following \citet{wu10}.
The last raw shows the model prediction for the relative magnitudes relative to image A, from the best-fit mass model.
}
\end{deluxetable}


\begin{figure}
\epsscale{1.1}
\plottwo{lens3.eps}{lens4.eps}
\caption{
PS1 images of the lens candidate. North is up and east is to the left. Left: close-up color image using the $g$ (blue), $i$ (green) and $y$ (red) filters. Right: $y$-band image of a 2 $\arcmin$ region around the system. The bright star south-east of the lens is saturated in the other bands.}
\label{lens}
\end{figure}



\begin{figure}
\epsscale{1.1}
\plottwo{redshiftseparate.eps}{redshifttogether.eps}
\caption{
Photometric redshift estimates using the method in \citet{wu10}. 
Left: individually for each image; right: for all images simultanesously.}
\label{redshift}
\end{figure}


\begin{figure}
\epsscale{1}
\plotone{hostlens.eps}
\caption{
Original images ($10\arcsec\times10\arcsec$, first row) and residuals after subtracting the best-fit model with Hostlens. From left to right: $g$, $r$, $i$, $z$, $y$-band. In the second raw, the lensing galaxy is not including in the modeling, whereas in the third raw it is. The images are in linear scale, and cover the full dynamic range.}
\label{hostlens}
\end{figure}


\begin{figure}
\epsscale{0.6}
\plotone{critcrv1.eps}
\caption{
Gravitational lens modelling. The blue and red lines show the critical curves and caustics, respectively. The green dot shows the quasar position in the source plane, and the black dots show the lens and the observed image positions.}
\label{model}
\end{figure}


\begin{figure}
\epsscale{1.0}
\plotone{model-photometry.pdf}
\caption{
Comparison between the observed quasar image magnitudes relative to image A (vertical colored lines) and the model predictions (blue histogram) for each PS1 band. The model predictions were drawn from a MCMC exploration of the range of models}
\label{modelphot}
\end{figure}

\begin{figure}
\epsscale{1.1}
\plotone{ilens_out_file_mcmc.png}
\caption{
Degeneracies between the parameters used to model the observed system in $i$-band with Hostlens. The three contour lines mark the 1-, 2- and 3-$\sigma$ limits. The vertical lines in the histograms mark the 16th, 50th and 84th percentiles. FWHMs are given in arcseconds, positions in pixels, and fluxes in counts.}
\label{mcmc_hostlens}
\end{figure}

\end{document}
